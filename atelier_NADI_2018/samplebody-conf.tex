\section{Introduction to Visualization in Machine Learning}

% + Intro AI 
Our lives are getting better and better thanks to the intelligent systems.
A computer vision system for analyzing medical images can support doctors to make diagnosis faster and more accurate.
A smart home equipped with voice controlled devices is made possible due to the innovation in voice recognition and human language understanding system.
They are some of numerous examples of learning system, the system can learn and improve itself from data, which can be million of medical images, voice records or texts in many languages.

% + Data is complex, high dimensional and the need of visualization.
These data are generally complex with a large number of features, attributes or characteristics.
The datasets are sometimes tied to a particular domain, thus analyzing and understanding them is a crucial task.
Representation learning~\cite{bengio2013representation} deals with the problem of representing data in form of numerical features used by the machine learning algorithms, but these representations are not understandable for human.
In contrast, visualization technique provides visual representations that summary the characteristics of the data to help people understand and carry out the analyzing task more effectively~\cite{munzner2014visualization}.
By using machine learning, more specifically, \emph{the dimensionality reduction} (DR) techniques, people can access a large amount of data for visually finding the interesting patterns.

% + The core algorithm to do viz is DR
Our works focus on DR techniques, which are \emph{unsupervised} learning methods that help to reduce the dimensionality of the data for processing and analyzing tasks.
When the data is reduced to two or three dimensions, they can be easily visualized in scatter plot in 2D or 3D space.
Moreover, pattern discovering in visualization is an interactive process in which the user can, for example, change the parameters of the DR algorithms and observe the change in the visualization immediately.
% + Intro our work: interactive DR for Viz
In addition, the visualization result is not always perfect since the algorithm can make mistakes or the visualization does not satisfy the user's viewpoint. To deal with this issue, the integration of human's feedback is necessary.
The main challenge in our research is how the machine learning algorithm can learn from the users' feedbacks through interactions to reveal their intend and produce the visualization that well responds to their needs.


\section{The Need of Human-in-the-loop}
A structured literature analysis about the visual interaction with dimensionality reduction by Sacha et. al.~\cite{Sacha2017Interaction} and the methods therein are highly recommended to understand the interactive process in visual analysis and visualization.
If the users have their analysis questions, they could simply choose the appropriate DR algorithms to produce the expected visualizations.
But when the analysis problem is complex and they do not have these questions beforehand, interacting with the visualizations will be a good idea to discover the data.
The prior knowledge of the domain expert or the interaction of end-user can be used to guide the DR algorithms, as shown in some following intuitive examples.

The \emph{forward and backward projection} technique~\cite{cavallo2017FWBW} allows user to move points in the scatter plot to see how the feature values of the corresponding data point in high dimensional space are changed. This helps user figure out which important features determine the position of the points in the visualization.
The work of Kim et. al. \cite{Kim2016InterAxis} shows an interactive visualization for an automobile dataset, in which the user can use examples to guide the algorithm to construct the meaningful coordinate axes.
More precisely, they can place a 2-seaters car on the left, a 7-seaters SUV on the right, an electric car on the top and a gasoline car on the bottom. The algorithm will automatically place the other cars from left to right by the increasing order of size and from top to bottom by the fuel technologies.
In summary, tight integration of machine learning algorithms, visualizations and user interactions is essential and that navigates our research direction.


\section{Works in Progress and Perspective}

% + Goal of our current works
Our current work is to integrate the user interaction into a dimensionality reduction method to steer the visualization.
In one direction, we let the users directly modify the visualization result by moving the points in the scatter plots.
These moved points play the role of guidance for the DR algorithm, from which a new visualization will be produced.
In other direction, we let the users define their requirements about the visualization they need and automatically find a visualization that well respects their expectations.

% + Interactive with t-SNE and PPCA
In the first approach, by allowing the users to interact directly with the scatter plot, we can translate their cognitive feedback in the form of moving points into constraints for optimization algorithm.
As a result with \emph{t-Distributed Stochastic Neighbor Embedding} (t-SNE), a neighborhood preserving DR technique, user can easily divide up a group or merge some groups in the visualization.
When the points are moved, their neighbors have a tendency to follow so they can joint together or separate to create new groups.
Another result with \emph{Probabilistic Principal Component Analysis} (PPCA), a global linear DR method, shows that the visualization is rotated in the same direction as the moved points to preserve the global visualization structure. That means the user can manually control the rotation of the visualization to find the useful meaning of two coordinate axes in the scatter plot.

% + User-steering viz with constraints.
In the second approach, we propose a visualization that matches the predefined requirements of the user.
In this way, we free the user from a tedious process of choosing the numeric parameters for a visualization technique (t-SNE in our case).
Instead, the user defines his requirements about the visualization in term of pairwise constraints between examples.
These constraints encode which points should be places close together or far apart in the embedded space.
Our algorithm translates these requirements into the \emph{constraint-preserving score} which is used as a criterion to choose the best visualization.

% + Perspective in the future: multi-aspect
However, the real world datasets are much more complicated, not only in term of the number of dimensions or data points but also of how they are interpreted.
The visualization for a dataset is thus not unique. Different people under different view points can understand the dataset in different ways and expect to see different visualizations.
For example, with the movie reviews dataset (containing the comments about movies in human-readable text), a typical visualization can highlight the groups of positive, negative or neutral comments.
Although this is a good way to discover the pattern in the text, the user still wants to discover more semantic patterns in the comments like emotions (funny, amusing or sad movies), movie types (sci-fi, comedy, drama, super-hero etc.)
Our perspective for the future work is to develop an interactive \emph{multi-aspects visualization} technique, in which the algorithm is expected to uncover all the hidden semantic aspects in the dataset.
This technique is also required to understand the user's interactions to show an appropriate visualization corresponding to the desired semantic aspect that the user is looking for.