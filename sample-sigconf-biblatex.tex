\documentclass[sigconf,natbib=false]{acmart}

\settopmatter{printacmref=false} % Removes citation information below abstract
\renewcommand\footnotetextcopyrightpermission[1]{} % removes footnote with conference information in first column
\pagestyle{plain} % removes running headers
% https://tex.stackexchange.com/questions/346292/how-to-remove-conference-information-from-the-acm-2017-sigconf-template

%drt24 hacks
% Letter paper
\setlength{\paperheight}{11in}
\setlength{\paperwidth}{8.5in}

% Hack to try to make acmart work with biblatex: https://tex.stackexchange.com/questions/37076/is-it-possible-to-load-biblatex-with-a-class-that-has-already-loaded-natbib
\let\citename\relax
\RequirePackage[abbreviate=true, dateabbrev=true, isbn=true, doi=true, urldate=comp, url=true, maxbibnames=9, backref=false, backend=biber, style=ACM-Reference-Format, language=american]{biblatex}

\addbibresource{mybib.bib}
\renewcommand{\bibfont}{\Small}

\usepackage{booktabs} % For formal tables

% Copyright
\setcopyright{none}
%\setcopyright{acmcopyright}
%\setcopyright{acmlicensed}
%\setcopyright{rightsretained}
%\setcopyright{usgov}
%\setcopyright{usgovmixed}
%\setcopyright{cagov}
%\setcopyright{cagovmixed}

\begin{document}
\title{Interactive Machine Learning for Visualization}

\author{Viet Minh Vu}
\email{vuvietminh@unamur.be}
\affiliation{\institution{Universit\'e de Namur - Faculty of Computer Science}} %  - PReCISE

\author{Beno\^it~Fr\'enay}
\email{benoit.frenay@unamur.be}
\affiliation{\institution{Universit\'e de Namur - Faculty of Computer Science}} %  - PReCISE

% The default list of authors is too long for headers}
\renewcommand{\shortauthors}{V. M. Vu}

\begin{abstract}
In this text, we briefly introduce a general context of the \emph{artificial intelligent} systems, then narrow down to the \emph{machine learning} system that can learn and improve itself from \emph{high dimensional} and complex data. Then we draw attention to the \emph{dimensionality reduction} methods for reducing the dimension of data in order to visualize them. We focus on the interactive, human-centered visualization process and present several approaches integrating human's feedback to steer the visualization. We end this text by our perspective of an \emph{multi-aspects} visualization technique that can help us understand the dataset under different points of view.

\end{abstract}

\keywords{Machine Learning, Visualization, User Interaction}

\maketitle

\section{Introduction to Visualization problem in Machine Learning}

% + Intro AI 
Our lives are getting better and better thanks to the intelligent systems.
A computer vision system for analyzing medical images can support the doctors to make the diagnosis faster and more accurate.
A smart home equipped with voice controlled devices is made possible due to the innovation in voice recognition and human language understanding system.
They are some of numerous examples of learning system, the system can learn and improve itself from data, which can be million of medical images, voice records or texts in many languages.

% + Data is complex, high dimensional and the need of visualization.
These data are generally complex with large numbers of features, attributes or characteristics.
The datasets are sometimes tied to a particular domain, thus analyzing and understanding them is a crucial task.
Representation learning [CITE] deals with the problem of representing data in form of numerical features used by the machine learning algorithms, but these representations are not understandable for human.
In contrast, visualization technique provides visual representations that summary the characteristics of the data to help people understand and carry out the analyzing task more effectively [CITE] .
By using the machine learning algorithm, more specific, \emph{the dimensionality reduction} (DR) technique, people can access a large amount of data for visually finding the interesting patterns.

% + The core algorithm to do viz is DR
Our works focus on DR techniques, which are the unsupervised machine learning methods that help to reduce the dimensionality of the data for processing and analyzing tasks.
When the data is reduced to two or three dimensions, they can be easily visualized in scatter plot in 2D or 3D space.
Moreover, pattern discovering in the visualization is an interactive process in which the user can for example change the parameters of the DR algorithms and observe the change in the visualization immediately.
% + Intro our work: interactive DR for Viz
In addition, the visualization result is not always perfect since the algorithm can make mistakes or the visualization does not satisfy the user's viewpoint. To deal with this issue, the integration of human's feedback is necessary.
The main challenge in our research is how the machine learning algorithm can learn from feedback of the users' interactions to reveal their intend and produce the visualization that well responds to their needs.

\section{The Need of Human-in-the-loop}

% + Algo can make mistakes.

% + Why human can help.

% + How to integrate human knowledge

% + State of the art human interaction method.

% \section{Integration of User Interactions into Visualization, the Showcases}

% + Scenario that human can help in visualization

% + Choose some (3) examples (very attractive methods)
The work of [CITE] show an interactive visualization for the car dataset, in which the user can use the examples to guide the algorithm to construct the meaningful coordinate axes.
For example, the user can place the small car of two places on the left, the large car on the right, the electric car on the top and the gas

\section{Works in Progress and Perspective}

% + Goal of our current works
Our current work is to integrate the user interaction into a dimensionality reduction method to steer the visualization.
In one direction, we let the users directly modify the visualization result by moving the points in the scatter plots.
From these moved points which play a role as the guides for the DR algorithm, a new visualization will be produced.
In other direction, we let user define their requirements about the visualization they need and automatically find a visualization that well respects these constraints.

% + Interactive with t-SNE and PPCA
In the first approach, by allowing the users to interact directly with the scatter plot, we can translate their cognitive feedback in form of moving points into the constraints for the optimization algorithm.
As a result with \emph{t-Distributed Stochastic Neighbor Embedding} (t-SNE), a neighborhood preserving DR technique, the user can easily divide up a group or merge some groups in the visualization.
When the points are moved, their neighbors have a tendency to follow so they can joint together or separate to create new groups.
Another result with \emph{Probabilistic Principal Component Analysis} (PPCA), a global linear DR method, shows that the visualization is rotated in the same direction with the moved points to preserve the whole global structure of the visualization. That means the user can manually control the rotation of the visualization to find the useful meaning of two coordinate axes in the scatter plot.

% + User-steering viz with constraints.
In the second approach, we propose a visualization that matches the predefined requirements of user.
In this way, we free user from a tedious process of choosing the numeric parameters for a visualization technique (t-SNE in our case).
Instead, the user defines his requirements about the visualization in term of pairwise constraints between examples.
These constraints encode which points should be places close together or far apart in the embedded space.
Our algorithm translates these requirements into a \emph{constraint-preserving score} which is used as a criterion to choose the best visualization.

% + Perspective in the future: multi-aspect
However, the real world datasets are much complicated not only in term of the number of dimensions or data points but also in term of how they are interpreted.
The visualization for a dataset is thus not unique. Different people under different view points can understand the dataset in different way and they expect to see different visualizations.
For example, with the movie reviews dataset containing the comments of users about movies (in human-readable text), a common visualization can highlight the groups of positive, negative or neutral comments.
Although this is a good way to discover the pattern in the text, the user still wants to discover more semantic patterns in the comments like emotions (funny, amusing or sad movies), movie types (sci-fi, comedy, drama, super-hero etc.)
Our perspective for the future work is to develop an interactive \emph{multi-aspects visualization} technique, in which the algorithm is expected to uncover all the hidden semantic aspects in the dataset.
This technique is also required to understand the user's interactions to show an appropriate visualization corresponding to the desired semantic aspect that the user is looking for.

\printbibliography

\end{document}
